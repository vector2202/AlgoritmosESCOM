%%Ambiente
\documentclass[12pt]{report}

%% Paquetes
\usepackage{hyperref}
\usepackage[spanish, mexico]{babel}
\usepackage{listings}

\setlength{\parindent}{0cm}

%%Titulo
\title{Practica 2}
%%Autor
\author{Los utlimos}

\maketitle
%%Inicio del documento
\begin{document}
%% Casos
\section*{Analsis de casos por codigo}
\subsection*{Busqueda con Arbol Binario de busqueda}
Operaciones basicas consideradas: comparacion entre la llave a buscar y el elemento del nodo, la comparacion entre el arbol auxiliar y nulo.\\
Condiciones del mejor caso: que el elemento a buscar este en la raiz.\\
Condiciones del peor caso: un arbol que tenga n niveles y el elemento este en el nodo hoja hoja.\\
Funcion de complejidad temporal del mejor caso $= 1$\\
Funcion de complejidad temporal del peor caso $= 3n$\\
Funcion de complejidad temporal del caso medio $= \frac{3n+1+3log_2(n)}{3}$
\subsection*{Busqueda Binaria}
Operaciones basicas consideradas: comparaciones de la llave con elementos del arreglo, actualizacion de limite inferior y superior.\\
Condiciones del mejor caso: cuando el numero a buscar esta en medio del arreglo.\\
Condiciones del peor caso: cuando el numero a buscar esta al principio o al final del arreglo.\\
Funcion de complejidad temporal del mejor caso $= 3$\\
Funcion de complejidad temporal del peor caso $= 2log_2(n)$\\
Funcion de complejidad temporal del caso medio $= \frac{2log_2(n) + 3}{2}-$
\subsection*{Busqueda exponencial}

\subsection*{Busqueda de fibonacci}

\subsection*{Busqueda lineal}
Operaciones consideradas: Comparaciones entre el elemento buscado y los elementos del arreglo.\\
Condición de mejor caso: El número buscado está en la primera posición. \\
Condiciones Peor caso: El número buscado está al final del areglo. \\
Función de complejidad temporal del mejor caso : 1. \\
Función complejidad temporal del peor caso: n. \\
Función complejidad caso medio: $\frac{1}{n}+(n-1)$.\\

\end{document}
